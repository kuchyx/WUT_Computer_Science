\documentclass[12pt]{report}

\usepackage{mathtools}
\usepackage{amsmath}
\usepackage{systeme}
\usepackage{siunitx}
\usepackage[T1]{fontenc}
\usepackage{hyperref}
\usepackage{bigfoot}
\usepackage[numbered, framed]{matlab-prettifier}
\usepackage{filecontents}
\usepackage{graphicx}
\hypersetup{
colorlinks,
citecolor=black,
filecolor=black,
linkcolor=black,
urlcolor=black
linkto=all,
}

\newenvironment{simplechar}{%
   \catcode`\^=12
}{}

\title{Numerical Methods, project C, Number 32}
\author{Krzysztof Rudnicki\\ Student number: 307585 \\ Advisor: dr hab. Piotr Marusak}
\date{\today}

\let\ph\mlplaceholder % shorter macro
\lstMakeShortInline"

\lstset{
  style              = Matlab-editor,
  basicstyle         = \mlttfamily,
  escapechar         = ",
  mlshowsectionrules = true,
}

\begin{document}




\maketitle
\tableofcontents

\chapter{Determine polynomial function fitting experimental data}
\section{Problem}
Given following samples:
\begin{center}
  \begin{tabular}{| c | c |}
\hline
$x_i$ & $y_i$ \\
\hline
-5 & -6.5743\\
\hline
-4 & 0.9765\\
\hline
-3 & 3.1026\\
\hline
-2 & 1.8572 \\
\hline
-1 & 1.3165 \\
\hline
0 & -0.6144 \\
\hline
1 & 0.1032 \\
\hline
2 & 0.3729 \\
\hline
3 & 2.5327 \\
\hline
4 & 7.3857 \\
\hline
5 & 9.4892 \\
\hline

\end{tabular}
\end{center}

We have to determine polynomial function $ y  = f(x) $ that best fits this data.
We will use least-square approximation using system of normal equation with QR factorization.

\section{Theoretical introduction}



\chapter{Determine trajectory of the motion}
\section{a) Runge-Kutta method of $4^{th}$ order and Adams PC}
\subsection{Problem}
We are given following equations:
\[ \frac{dx_1}{dt} = x_2 + x_1(0.5 - x_1^2 - x_2^2) \]
\[ \frac{dx_2}{dt} = -x_1 + x_2(0.5 - x_1^2 - x_2^2) \]

And we have to determine the trajectory of the motion on interval $[0, 15]$ with following initial conditions:
$ x_1(0) = 8; x_2(0) = 9 $
In this section we will use Runge-Kutta method of $4^{th}$ order and Adams PC with different step-sizes until we find an optimal constant step size - when the decrease of the step size does not influence the solution significantly.

\subsection{Theoretical Introdution}

\section{b) Runge-Kutta method of $4^{th}$ order with variable step size automatically adjusted}
\subsection{Problem}
We are given following equations:
\[ \frac{dx_1}{dt} = x_2 + x_1(0.5 - x_1^2 - x_2^2) \]
\[ \frac{dx_2}{dt} = -x_1 + x_2(0.5 - x_1^2 - x_2^2) \]
And we have to determine the trajectory of the motion on interval $[0, 15]$ with following initial conditions:
$ x_1(0) = 8; x_2(0) = 9 $
In this section we will use Runge-Kutta method of $4^{th}$ order with step size automatically adjusted by the algorithm, with error estimation made according to the step-doubling rule.
\subsection{Theoretical Introdution}

\begin{thebibliography}{9}
\bibitem{texbook}
Piotr Tatjewski (2014) \emph{Numerical Methods}, Oficyna Wydawnicza Politechniki Warszawskiej
\end{thebibliography}


\end{document}

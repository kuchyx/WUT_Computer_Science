\documentclass[12pt]{article}

\usepackage{listings}
\usepackage{graphicx}

\lstset{
  basicstyle=\small,
  breaklines=true
  }


\date{\today}
\title{ECOTE - Final Documentation \\ 
Translator of a \LaTeX \, subset to HTML
}
\author{Krzysztof Rudnicki, 307585 \\
Semester: 2023L}

\begin{document}
\maketitle
\section{General overview and assumptions}
My task was to create a translator of \LaTeX \, subset to selected text format with focus on \LaTeX \, tables \\ 
I decided to change to translator of \LaTeX \, subset to HTML since I know \LaTeX \, very well and HTML relatively well, I decide to translate \LaTeX into HTML since HTML is easy, a little bit different than \LaTeX and popular which makes this translator a practical tool.
\subsection{Assumptions}
\begin{itemize}
    \item "Tables" will be represented using \LaTeX \, \emph{table} environment 
    \item All changes influenced by different \LaTeX \, document class are ignored in html 
    \item Everything between documentclass and begin document is ignored when generating html
    \item Program tries to generate final html even if encountering errors which would make final html not complete
    \item Only commands defined in \LaTeX \, subset are translated, rest is translated literally as raw text to html
\end{itemize}
\section{Functional requirements}
The goal of the project is to transform .tex file to (working ) .html file if the subset of .tex file is within project scope or output error message explaining why the html could not be outputed
\subsection{\LaTeX \, subset}
This project will focus almost exclusively on \emph{tabular} environment \\
\begin{itemize}
    \item $\backslash$documentclass\{class\}: Defines what layout standard \LaTeX will use 
    \item $\backslash$begin\{document\}: Ends (in our case empty) preamble
    \item $\backslash$end\{document\}: Ends \LaTeX \, document
    \item $\backslash$begin\{tabular\}[pos]\{table spec\}: Opens environment used to typeset tables 
    \item $\backslash$end\{tabular\}: Closes environment used to typeset tables 
\end{itemize}
Supported tabular arguments:
\begin{itemize}
    \item l, c, r - respectively left-justified, centered and right-justified column
    \item  | - single vertical line 
    \item  || - double vertical line 
    \item p\{'width'\} - paragraph column (that wraps), aligned at the top 
    \item m\{'width'\} - paragraph column (that wraps), aligned at the middle 
    \item b\{'width'\} - paragraph column (that wraps), aligned at the bottom 
\end{itemize}

Supported commands inside tabular environment
\begin{itemize}
    \item \& - separate columns
    \item $\backslash$$\backslash$ - start new row
    \item $\backslash$hline - horizontal line
    \item $\backslash$cline\{i-j\} horizontal line beginning in column i and ending in column j
    \item $\backslash$newline - new line \emph{inside} the cell
\end{itemize}
\section{Implementation}
I decided to use Python as a language in which I will implement my solution \\ 
The reasons for using python are as follow:
\begin{enumerate}
    \item It is the easiest language among those that I know
    \item I know it enough to be confident in my ability to implement this solution in python
    \item I want to learn python more through this project
\end{enumerate}
Negative aspects of python which is that it is very slow language do not bother me as I believe the project scope will not be big enough for this to become an issue
  
\subsection{General architecture}
\begin{tabular}{|l|p{10cm}|}
    \hline 
    Module & Description \\ 
    \hline 
    Main & Handles texfile, puts it to a function translating to html and saves output html file \\
    \hline
    FileHandler & Handles reading latex file and transforming it to string in python  \\ 
    \hline 
    ReadStart & Checks documentclass and begin document functions of latex file and if they are correct outputs starting html tags \\
    \hline
    HandleInside & Goes through inside of LaTeX document and transforms each of its part to HTML
    \\
    \hline
    HandleTable & Reads table, including table parameters and actual contents and transforms it to html \\ 
    \hline
    TabInsideHandler & Handles actual content of table and translates them to html \\
\end{tabular}






\subsection{Data structures}
\begin{itemize}
\item File entered by user is represented by python File class 
\item Parsed \LaTeX \, code is represented by data structure based on node classes 
\item Tabular environment parameters are stored in an array, if the parameter contains additional optional parameters they are stored in a pair of the parent argument and array of all optional arguments 
\item Generated HTML code is stored in node classes 
\item Final HTML code is stored in plain text and then written to File
\end{itemize}

\subsection{Module descriptions}
\subsubsection{Main}
Puts latex filename to filehandler, sends retrieved string to function translating tex to html and then saves final html file 
\subsubsection{FileHandler}
Handles texfile and transforms it to python string \\ 
Input: \\ 
tex\_filename - LaTeX file filename \\ 
Functions: 
\begin{itemize}
    \item Handle arguments - Checks if arguments are correct, reads file and saves it to python string
    \item Invoke Latex Handler - Sends file from file handler to LatexHandler 
\end{itemize}
\subsubsection{LatexHandler}
Transforms tex string to html
Input: tex string
Functions: 
\begin{itemize}
    \item ReadStart - Checks begininig of tex file and returns html code if the begining is correct
    \item HandleInside - Handles insides of tex file, both tables and loose text and translates it to html
\end{itemize}
\subsubsection{ReadStart}
Checks begininig of tex file and returns html code if the begining is correct \\ 
Input: tex string \\ 
Output: html\_string

\begin{itemize}
\item read\_document\_class - Checks if document class exists and is written according to tex rules, returns <!DOCTYPE html> if yes
\item read\_begin\_document - (optional) Checks if begin\{document\} function was written in tex file if yes adds <html> to html\_string
\item HandleInside - Passes parsed tex data and html\_string to function handling inside of tex document 
\end{itemize}

\subsubsection{HandleInside}
Translates tex string starting at begin document and ending at end document, returns final html\_string \\ 
Parameters: \\ 
\begin{itemize}
\item tex\_data - String read from tex file
\item html\_string - Final output string to which table data and loose text will be added 
\end{itemize}
Functions: 
\begin{itemize}
    \item HandleTable - Converts tex tables to html tables
    \item HandleLoose - Goes through text that is not in table and converts it to html  
\end{itemize}

\subsubsection{HandleTable}
Converts tex tables to html tables \\ 
Parameters: \\ 
\begin{itemize}
\item tex\_data - String read from tex file
\item html\_string - Final output string to which table data and loose text will be added 
\end{itemize}
Functions: 
\begin{itemize}
    \item TabArgHandler - Reads tex table parameters and translates them to html style
    \item TabInsideHandler - Reads inside of tex table and together with parse parameters outputs correct table layout and look
\end{itemize}

\subsubsection{TabInsideHandler}
Together with html table parameters translates inside of tex table to html  \\ 
Parameters: \\ 
\begin{itemize}
\item TableContents - Latex string containing only inside of table
\end{itemize}
Functions: 
\begin{itemize}
    \item ReadTableContent - Parses table content and checks for errors 
    \item ConvertToHTML - Converts table inside to html
\end{itemize}

\subsubsection{TabArgHandler}
Reads tex table parameters and translates them to html style  \\ 
Parameters: \\ 
\begin{itemize}
\item TableArguments - Arguments provided with $\backslash$begin\{tabular\} environment
\end{itemize}
Functions: 
\begin{itemize}
    \item ReadTableARguments - Parses table arguments and checks for errors 
    \item ConvertToHTML - Converts table arguments to html style
\end{itemize}

\subsection{Input/output description}
Input is a .tex file (\LaTeX \, file) \\ 
Outpus is an .html file \\ 
In case of errors error message will be outputed on the terminal \\
Input File path is entered as an argument to terminal with "-i" or "--input" flag for example:
\begin{lstlisting}[language=bash]
python main.py -i texFile.tex
\end{lstlisting}
Output file path can be named by user by using "-o" or "--output" flag:
\begin{lstlisting}[language=bash]
python main.py -i texFile.tex -o htmlFile.html
\end{lstlisting}
If no "-o" flag is issued the output file will have the same name as input file with changed extension to html (so in this example texFile.tex will become texFile.html) \\ 
If the path to file name consists of spaces, path name needs to be but in ""
\begin{lstlisting}[language=bash]
python main.py -i "My Folder/input.tex"
\end{lstlisting}
\subsection{Others}
\section{Unit tests}
There are over 12 unit tests  for program functions, all of them can be ran by using python test environment:
\begin{lstlisting}
python -m pytest unit_tests

collected 12 items                                                                                                            

unit_tests/test_code/test_begin_document.py .                                                                           [  8%]
unit_tests/test_code/test_begin_tabular.py .                                                                            [ 16%]
unit_tests/test_code/test_document_class.py .                                                                           [ 25%]
unit_tests/test_code/test_length_conversions.py .                                                                       [ 33%]
unit_tests/test_code/test_only_pipes_and_space.py .                                                                     [ 41%]
unit_tests/test_code/test_split_columns.py .                                                                            [ 50%]
unit_tests/test_code/test_split_rows.py .                                                                               [ 58%]
unit_tests/test_code/test_tabular_columns_parameters.py .                                                               [ 66%]
unit_tests/test_code/test_tabular_parameters.py .                                                                       [ 75%]
unit_tests/test_code/test_tabular_required_parameters.py .                                                              [ 83%]
unit_tests/test_code/test_translate_column.py .                                                                         [ 91%]
unit_tests/test_code/test_translate_inside_to_html.py .                                                                 [100%]

===================================================== 12 passed in 0.02s ======================================================
\end{lstlisting}
\section{Functional test cases}

\begin{tabular}{|p{3cm}|p{6cm}|p{6cm}|}
    \hline 
    Title  &  Input (\LaTeX) & Output \\
    \hline

    empty file &
    \begin{lstlisting}

    \end{lstlisting}&
    \begin{lstlisting}
Error! expected \documentclass at the begining of LaTeX file 
    \end{lstlisting} \\
    \hline

    Document class &
    \begin{lstlisting}
\documentclass[options]{class}
    \end{lstlisting}&
    \begin{lstlisting}
Error! expected \begin{document} after document class
    \end{lstlisting} \\
    \hline

    \begin{lstlisting}
Extra text between document class and begin document
    \end{lstlisting}&
    \begin{lstlisting}
\documentclass[options]{class}
"extra text"
\begin{document}
    \end{lstlisting}&
    \begin{lstlisting}
Error! unexpected text between document class and begin document
    \end{lstlisting}  \\
    \hline 

\begin{lstlisting}
Just document class and begin document
\end{lstlisting}&
\begin{lstlisting}
\documentclass[options]{class}
\begin{document}
\end{lstlisting}&
\begin{lstlisting}
Error! no \end{document} at the end of LaTeX code
\end{lstlisting}  \\
\hline 

    \end{tabular} 

\begin{tabular}{|p{3cm}|p{6cm}|p{6cm}|}
\hline 
Title  &  Input (\LaTeX) & Output \\
\hline

\begin{lstlisting}
Just document class and begin/end document
\end{lstlisting}&
\begin{lstlisting}
\documentclass[options]{class}
\begin{document}
\end{document}
\end{lstlisting}&
\begin{lstlisting}
<html>
</html>
\end{lstlisting}  \\
\hline 

\begin{lstlisting}
Plain text inside
\end{lstlisting}&
\begin{lstlisting}
\documentclass[options]{class}
\begin{document}
Lorem ipsum dolor sit amet.
\end{document}
\end{lstlisting}&
\begin{lstlisting}
<html>
Lorem ipsum dolor sit amet.
</html>
\end{lstlisting}  \\
\hline 

\begin{lstlisting}
Reduntant \end{document} (ignored)
\end{lstlisting}&
\begin{lstlisting}
\documentclass[options]{class}
\begin{document}
Lorem ipsum dolor sit amet.
\end{document}
\end{document}
\end{lstlisting}&
\begin{lstlisting}
<html>
Lorem ipsum dolor sit amet.
</html>
\end{lstlisting}  \\
\hline 

\begin{lstlisting}
LaTeX comments
\end{lstlisting}&
\begin{lstlisting}
\documentclass[options]{class}
\begin{document}
Lorem ipsum dolor sit amet.
% some comment
\end{document}
\end{document}
\end{lstlisting}&
\begin{lstlisting}
Error! LaTeX comment detected at line 3
\end{lstlisting}  \\
\hline 



\end{tabular}

\begin{tabular}{|p{3cm}|p{6cm}|p{6cm}|}
    \hline
Title  &  Input (\LaTeX) & Output \\
\hline

\begin{lstlisting}
Table with vertical lines
\end{lstlisting}&
\begin{lstlisting}
\documentclass[options]{class}
\begin{document}
\begin{tabular}{ l | c | r }
test & 2 & test \\
4 & 5 & 6 \\
\end{tabular}
\end{document}
\end{lstlisting}&
\begin{lstlisting}
<html>
<table>
<tr>
<td align='left'>test</td>
<td align='center' style="border-left: 1px solid black;">2</td>
<td align='right'>test</td>
</tr>
<tr>
<td align='left'>4</td>
<td align='center' style="border-left: 1px solid black;">5</td>
<td align='right'>6</td>
</tr>
</table>
</html>
\end{lstlisting}  \\
\hline 

\begin{lstlisting}
Missing &
\end{lstlisting}&
\begin{lstlisting}
\documentclass[options]{class}
\begin{document}
\begin{tabular}{ l c r }
1 & 2 & 3 \\
4 & 5 6 \\
\end{tabular}
\end{document}
\end{lstlisting}&
\begin{lstlisting}
Error! Missing third column in second row 
\end{lstlisting}  \\
\hline 

\end{tabular}

\begin{tabular}{|p{3cm}|p{6cm}|p{6cm}|}
    \hline
Title  &  Input (\LaTeX) & Output \\
\hline

\begin{lstlisting}
Too much columns
\end{lstlisting}&
\begin{lstlisting}
\documentclass[options]{class}
\begin{document}
\begin{tabular}{ l c r }
1 & 2 & 3 & 4 & 5 \\
4 & 5 6 \\
\end{tabular}
\end{document}
\end{lstlisting}&
\begin{lstlisting}
Error! Too much columns in row 1, expected 3, got 5
\end{lstlisting}  \\
\hline 

\end{tabular}
    
\end{document}